% Mengubah keterangan `Abstract` ke bahasa indonesia.
% Hapus bagian ini untuk mengembalikan ke format awal.
% \renewcommand\abstractname{Abstrak}

\begin{abstract}

  % Ubah paragraf berikut sesuai dengan abstrak dari penelitian.
  \emph{In Soccer Robotics Competition, IRIS team archieved 3rd  
  Position in RoboCup. In the game, IRIS Robots used Omnivision to 
  sensing their environtment. The current Calibration method is using 
  polynomial regression for one direction, so that the other direction 
  is not calibrated and give incorrect data. This Final Project propose 
  new method that use Machine Learning. The result shows that Omnivision 
  calibration using Machine Learning is better than using polynomial.}


\end{abstract}

% Mengubah keterangan `Index terms` ke bahasa indonesia.
% Hapus bagian ini untuk mengembalikan ke format awal.
% \renewcommand\IEEEkeywordsname{Kata kunci}

\begin{IEEEkeywords}

  % Ubah kata-kata berikut sesuai dengan kata kunci dari penelitian.
  \emph{Omnivision}, \emph{Calibration}, \emph{IRIS}

\end{IEEEkeywords}
