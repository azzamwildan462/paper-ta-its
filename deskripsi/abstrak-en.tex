% Mengubah keterangan `Abstract` ke bahasa indonesia.
% Hapus bagian ini untuk mengembalikan ke format awal.
% \renewcommand\abstractname{Abstrak}

\begin{abstract}

  % Ubah paragraf berikut sesuai dengan abstrak dari penelitian.
  \emph{Mobile Robot is a robot that can easily move. The movement of the robot can cause the camera angle to shift. This shift can be caused by the wrong manufacturing of the camera installation or a collision on the robot. The camera angle shift will cause the camera's interpretation of the outside world to be wrong. The use of Machine Learning methods in Omnivision camera calibration can correct the wrong camera interpretation without being influenced by the Omnivision camera manufacturing and installation process. The Machine Learning used is a Multi Layer Perceptron Neural Network with an activation function in the form of a sigmoid. The results of Machine Learning will be converted into a Lookup Table which will be used in the vision computation process of the robot. This method is better than the old polynomial regression method. This can be seen from the accuracy and precision produced by the Machine Learning method which is better than the polynomial regression method. The accuracy error of the Machine Learning method is 10.84 cm while the polynomial regression method is 20.77 cm. The precision error of the Machine Learning method is 1.20 cm and 4.10 cm while the polynomial regression method is 10.01 cm and 11.32 cm. By using the Machine Learning method in Omnivision camera calibration, the robot can move better.}


\end{abstract}

% Mengubah keterangan `Index terms` ke bahasa indonesia.
% Hapus bagian ini untuk mengembalikan ke format awal.
% \renewcommand\IEEEkeywordsname{Kata kunci}

\begin{IEEEkeywords}

  % Ubah kata-kata berikut sesuai dengan kata kunci dari penelitian.
  \emph{Omnivision}, \emph{Calibration}, \emph{IRIS}

\end{IEEEkeywords}
