% % Ubah judul dan label berikut sesuai dengan yang diinginkan.
% \section{Kesimpulan}
% \label{sec:kesimpulan}

% % Ubah paragraf-paragraf pada bagian ini sesuai dengan yang diinginkan.

% Dari hasil penelitian yang telah dilakukan, dapat disimpulkan bahwa metode kalibrasi kamera omnivision menggunakan \emph{Machine Learning} lebih baik daripada metode kalibrasi kamera omnivision menggunakan regresi polinomial. Hal ini dapat dilihat dari kemampuan metode kalibrasi kamera omnivision menggunakan \emph{Machine Learning} yang dapat meng-kalibrasi kamera omnivision pada semua arah. Sedangkan metode kalibrasi kamera omnivision menggunakan regresi polinomial hanya dapat meng-kalibrasi kamera omnivision pada satu arah saja. Selain itu, metode kalibrasi kamera omnivision menggunakan \emph{Machine Learning} juga membutuhkan waktu eksekusi yang lebih singkat daripada metode kalibrasi kamera omnivision menggunakan regresi polinomial. 

\section{Conclusion}
\label{sec:conclusion}

From the research that has been done, it can be concluded that the omnivision camera calibration method using Machine Learning is better than the omnivision camera calibration method using polynomial regression. This can be seen from the ability of the omnivision camera calibration method using Machine Learning to calibrate the omnivision camera in all directions. While the omnivision camera calibration method using polynomial regression can only calibrate the omnivision camera in one direction. In addition, the omnivision camera calibration method using Machine Learning also requires a shorter execution time than the omnivision camera calibration method using polynomial regression.